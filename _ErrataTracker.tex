Errata list / things to fix for Version 5.

%
1. Constant appearing in 6.5, #17. It's an integration problem, so it isn't technically *wrong*.
% Fixed 8/19/18

2. In Section 8.2, n^th partial sums need better definition. The formula implies that
S_n will add up terms r^0 + r + r^2 + ... + r^n, but this actually adds up n+1 terms. 
The geometric series portion should be rewritten. Not really used in the text, but it
influences answers at the back. (See email from Chuck Cusack.)

3. Page 6, line 3: refers to $\sin(x)/x$ when it should refer to $\sin(1/x)$.
% Fixed 8/20/18

pA.1 1.1#17: In the solution, the limit should seem to be exactly $-7$.
% Fixed 9/15/18


p74 02_01_exset_05: the enum counter resets (you \setcounter the wrong enum level?). also, inequalities 4 & 5 should be switched.
% Fixed 9/15/18    Used things like \item[(a)] to hard code the counter. Changed the > & < of parts (d) and (e) to match (a) and (b); answers already assumed that was the case.


p74 2.1#34 solution: g’=0 at x=0
% Fixed 9/15/18  Added $x=0$ to solution


pA.4 2.1#37: The solution should be negated.
% Fixed 9/15/18  Answer is now $-24$.

p82 2.3 para 2 line 2: “linear function” should be pluralized
% Fixed 9/15/18


p98 2.4#24: The solution should not have parts (b) or (c) (and no label ``(a)''). (I don’t think we added that in, but it’s not easy to check through your source for the exercises.)
% Fixed 9/15/18   Looks like this problem used to be used for a ``take the derivative two ways''
%  type problem, and now it is just a ``take the derivative'' problem.


p98 2.4#34: The solution should use $h’$, not $f’$.
% fixed 9/15/18, uses h'(t) instead of f'(x)


pA.6 2.6#29b solution: $+\sqrt[4]{108}$ should be $-\sqrt[4]{108}$.
% fixed 9/15/18  changes made 


p141 3.2 #3-10: In order to match Rolle's Theorem, the final $[a,b]$ should be $(a,b)$.
% fixed 9/15/18  changed to $(a,b)$


pA.6 3.2#23: 0 should not be included in the solution.
% 9/15/18 0 is removed


p173 4.1#15: The given problem has a unique solution at $x=0$.  The given solution works for $g(x)=\cos x+1$.
% 9/15/18  changed problem to $\cos(x) +1$ to match solution; probably initial intent


p179 4.2#4: ``circular'' should be ``spherical''.
% 9/15/18  yup.


p194 4.4#30 you probably want y instead of f(x)
%9/15/18  yup


pA.9 5.2#23 solution: The answer should be 22.
% fixed before 9/15; Troy?


pA.9 5.2#25 solution: The answer should be 0.
% fixed before 9/15; Troy?


p258 Example 5.5.8, Figure 5.5.14: Since you measured every 30 seconds, the time column should be labeled “Measurement #”.  Or you can leave it as time, but, e.g., change 7 to 3.5 min.
% 9/15/18  The label was changed to ``Time\\ (min)'' and the data is now time in minutes. 
% The text for the example was also altered to reflect the new data form.


p281 6.1#36 solution: For $\abs x<1$, we need $\frac12 \ln\abs{\ln \left(x^2\right)}+C$ (absolute values).
% 9/15/18   Changed the solution to $\frac{1}{2} \ln \left|\ln \left( x^2\right)\right|+C$ instead of worrying about when the inner \ln could be negative. More consistent with the integration step anyway


pA.11 6.4#23 solution: $\tan^{-1}\left(\frac{x+2}2\right)$ should be $\tan^{-1}\left(\frac{x+2}3\right)$.
% 9/15/18  fixed


pA.12 6.7#13 Solution: Should be $2x\sech^2(x^2)$.
% 9/15/18  actually it is 6.6 #13. And fixed from sec to sech.


pA.12 6.8#3 Solution: Should have $\le10$ instead of $<10$.
% 9/15/18  Yes, changed to \leq 10


p361 7.1#27-30: ``area triangle'' should be ``area of the triangle''.
% 9/15/18  fixed


p361 7.1#29: The solution should be 0, or a $(3,3)$ should become $(-1,3)$.
% 9/15/18  changed third point to (-1,3)


p361 7.1#30 solution: The area should be $\frac{21}2$.
% 9/15/18  yes   $21/2$


p388 Example 7.5.1 solution line 2: ``it is'' should be ``is it''.
% 9/15/1 fixed


pA.13 7.5#5b solution: The answer is a length, so the units should just be ft.
% 9/15/18  yes; it was ft-lb


p417 8.1#13-16: $a_n$ is multiply defined
% 9/15/18 Fixed; instead of each being a_n, they are now d_n


p433 8.2#34a solution: should have the negative with $n$ odd, not $n$ even.
% 9/15/18  Yes


p433 8.2#36 solution: the series started at $n=1$, so it should be (a) $S_n=\dfrac{e^{-1}-e^{-n-1}}{1-e^{-1}}=\dfrac{1-e^{-n}}{e-1}$, (b) $\dfrac{e^{-1}}{1-e^{-1}}=\dfrac1{e-1}$.
% 9/15/18 Changed to sum starting with n=0. Makes more sense, instead of making it even harder.


p433 8.2#38a solution: $a_n = \frac12\left(\frac{1}{2n-1}-\frac{1}{2n+1}\right)$, so that $S_n=\frac12\left(1-\frac1{2n+1}\right) = \frac{n}{2n+1}$.
% 9/16/18 think this refers to #40a, not 38. Answer is correct in text.
% Wonder if suggestion is really about adding one more step between
% S_n and final answer.


p441 8.3#22 solution: should specify $n\geq 3$ at the end, not $n\geq 2$.
% 9/16/18  Yes, changed 2 to 3.


pA.15 8.3#41a solution: $a_n/n<n$ should be $a_n/n<a_n$
% 9/16/18  Yes, changed n to a_n.


pA.15 8.4#21 solution: $e^2$ should be $e^{-2}$.
% 9/16/18  Yes.


p483 8.7#29-34 might want to ask for the $x^n$ term, since Taylor polynomials start with the 0th term.
% REVISIT  Will look into other books' usage. May specify within book what the terms are.


pA.17 8.8#11 solution: The Taylor Series should be $\frac12+\sum_{n=1}^\infty$.
% 9/16/18  Fixed


pA.19 9.2#45 solution: should be $t=k\pi$ for integral $k$.
% 9/16/18  Fixed


pA.19 9.3#11 solution: a: The second term in the numerator should be doubled; b: The normal line is $y=-x$.
% 9/16/18 Fixed


pA.19 9.3#29 solution: The intervals should be: concave up: $(-\pi/2,0)$; concave down: $(0,\pi/2)$.
% 9/16/18 fixed


p545 9.4#50 solution: $P(-1/2,\pi/3)$ should be $P(-1/2,2\pi/3)$
% 9/16/18


pA.20 9.4#53 solution: the origin is also a point of intersection
% 9/16/18  added P(0,0) = P(0,\pi/6)$ as point of intersection


pA.34 14.3#17c solution: -1t should be t-1
% 9/16/18   fixed


p869 14.3#20d solution: 250 should be 0
% 9/16/18 fixed


pA.46 In Algebra / Factoring by Grouping, $(cs+d)$ should be $(cx+d)$.
% 9/16/18 already fixed


pA.47 In Algebra / Taylor Series: should end with $+ . . .$ and pA.47 In Algebra / Maclaurin Series: same
% 9/16/18 both fixed, plus other uses of ``...'' instead of ``\cdots'' on same page

%%%%
%%%%  Revisit these
%%%%  
You mentioned that all exercise sets now start with an odd and end with an even.  There are a few sets with an odd number of problems:
01_04_exset_03
02_06_exset_02
04_04_exset_03
05_01_exset_01
05_01_exset_02
06_02_exset_03
06_03_exset_02
07_04_exset_04
08_07_exset_06
09_02_exset_09
10_04_exset_01
10_05_exset_05

also:
2.3#26 doesn’t seem to belong to that instruction set, leaving the set with an odd number of problems
7.1#19,20: These don’t really belong to their exercise set
10.2#21 throws off the parity
%%%%
%%%%  Revist above
%%%%

Grammar:
p423 8.2 just before Thm 8.2.2: $p$ is not in math mode
p496 8.8#12 solution: an open quote is never closed
p793 Def&Thm 13.4.2: (x_2,y_2) should have a comma after it
% 9/17/18 All three of above fixed


Misspellings.  The good news is I figured out how to spell check in my editor (and from the command line:
cat text/*tex | aspell list -t --ignore=3 --ignore-case | sort | uniq > misspell.txt
The bad news is I figured out how to spell check.

Multiple places: The units Newtons and Joules should be capitalized
% 9/17/18 Pretty sure that no, they are not. Look at Wikipedia entry on Newton.
% Removed a few cases at first of capitalized Newton, used \emph{newton} and
% \emph{joule} for their first use.

p292 6.2#4: original is misspelled.
p557 9.5#34 solution: Pythagorean is misspelled
p583 Example 10.2.6 solution para -2 line -3: unknown is misspelled
p607 Example 10.4.6 statement line 1: parallelepiped is missing its second “le”
p698 12.2 line -1: Multivariable is misspelled
p723 12.5 Examples 2, 4, 5 title: Multivariable is misspelled
p788 Definition 13.4.1: Variable is misspelled
p829 13.7 Figure 13.7.2 caption: canonical is misspelled
p841 14.1 para before Definition 14.1.1 line 3: definition is misspelled
p851 14.2 para 3 line -2: circle is misspelled
p865 para 2 line -3: continuously is misspelled
p884 Example 14.5.5 title: cylindrical is misspelled
p895 Example 14.6.3 statement line -2: positive is misspelled
% 9/17/18  All of the above is fixed

You tend to omit the second e in parameterize (and its relatives).  You can try searching for “arametriz”:
Section 9.2
Section 9.3
Section 10.5 x2
Section 11.1
Section 11.5 x5
Section 12.7
Section 14.1 x13 (plus two more commented out)
Section 14.3 x20
Section 14.4 x12
Section 14.5 x44 (I think)
Section 14.6 x6
Section 14.7 x8

solution to 12.2.20 b as -1, I think it should be 1.

I had forgotten a subtle point about l'Hopital's rule that is not mentioned in the book that stumped me for a while. I was trying to take the \lim x-->\infty (x^2+sin x)/x^2 as a student might (l'Hopital instead of algebra) and was confused when after the second application it failed to exist. I had forgotten that l'Hopital's rule only applies if the limit on the right exists. This is not stated in book and that can certainly cause misunderstandings since a student could draw the wrong conclusion (that the original limit does not exist). I would suggest maybe adding a phrase such as "assuming the limit on the right exists" in the two theorems.

6.5 #17 superfluous constant in answer

In page 136, problem 14: The graph and the given function do not match! Please fix this in the next version!  Yes, the inside function is supposed to be x^4 - 2x^2+1. Somehow that mistake has lived on for a very long time.

6.4 #23 has (x+2)/2 in the arctan, but it should be (x+2)/3.

5.2. Number 23 and 25 answers in the back are off. 

The solutions to 8.2 #31, 35, and 37 have r^n, but Theorem 8.2.1 has r^{n+1} , so one of these appears to be incorrect. I think the problem is Theorem 8.2.1 since the sum of the first n terms would be the sum from 0 to n-1, making r^n correct.


6, line 3, it is said that the function in figure 1.1.13(c) is sin(x)/x when the function is sin(1/x).

01_Limits_Involving_Infinity, line 238: I've added the following 
modifier at the end of the line: "for any positive integer $n$" since 
the statement is false for n less than or equal to zero.

03_Extreme_Values: when defining critical point, the text suggests that 
critical number and critical value are synonymous. Maybe I've done too 
much differential topology/geometry, but I consider critical number to 
be the x coordinate, while the critical value (i.e. function value) is 
the y coordinate.

03_Mean_Value_Theorem, line 104: there's a bit of awkwardness with the 
parenthetical remark and the preposition: "parallel with" makes sense, 
but between these two words there's (i.e. have the same slope), and I 
expect the next word to be 'as' rather than 'with'.
I've changed it to "parallel with (i.e. have the same slope as) the 
dashed line" in my version.

05_Riemann_Sums, line 59: "these rectangle" (missing an s at the end)

05_Definite_Integrals, line 144: "viewed a merely a" (first 'a' should 
be 'as')

06_Trigonometric integrals, line 152: "vise-versa" rather than vice-versa

07_Work, line 143: "peformed"

07_Work, line 199: "empyting"

10_Vector_Introduction, line 203: "nonvector"

10_Vector_Introduction, line 300: "unkown"

10_Vector_Introduction, line 388: "text" should be "texts"

11_Vector_Tangent_Normal, line 79: reference fig:tannorm3 (in the 
figure) should be ex_tannorm3

13_Center_of_Mass, line 39: "Vairable"

13_Cylindrical_Spherical, line 3: "desribing"

13_Cylindrical_Spherical, line 51: "canoncial"

14_Line_Integral_Intro, line 84: "defintion"

14_Line_Integral_Vector, line 163: "continously"

14_Vector_Fields, lines 156 and 181: the figures refer to parts (a), 
(b),... of the Example, but the parts are numbered 1, 2,... in the 
Example itself.

14_Parameterized_Surfaces, line 158: "cylinderical"

14_Surface_Integral, line 131: "postive"
